
%%%%%%%%%%%%%%%%%%%%%%%%%%%%%%%%%%%%%%%%%%
%%%%%%%%%%%%%                 %%%%%%%%%%%%
%%%%%%%%%%%%%    EXERCISE 1   %%%%%%%%%%%%
%%%%%%%%%%%%%                 %%%%%%%%%%%%
%%%%%%%%%%%%%%%%%%%%%%%%%%%%%%%%%%%%%%%%%%
\begin{exercise}[]{Prove that $\log (\log n) = o(n^k)$, where k is a positive constant. (ps: $\log n$ refers to $\log_2 n$.)}
  \begin{proof}
  \begin{equation}
    \lim _{n\rightarrow\infty} \frac{\log(\log n)}{n^k} = \lim _{n\rightarrow\infty} \frac{\frac{1}{n\ln n \cdot \ln 2}}{k n^{k-1}} = \lim _{n\rightarrow\infty} \frac{1}{kn^k \ln n \ln 2} = 0 < \infty
  \end{equation}
  Hence, by definition, $\log (\log n) = O(n^k)$.
  \end{proof}
  \label{ex1}
\end{exercise}

%%%%%%%%%%%%%%%%%%%%%%%%%%%%%%%%%%%%%%%%%%
%%%%%%%%%%%%%                 %%%%%%%%%%%%
%%%%%%%%%%%%%    EXERCISE 2   %%%%%%%%%%%%
%%%%%%%%%%%%%                 %%%%%%%%%%%%
%%%%%%%%%%%%%%%%%%%%%%%%%%%%%%%%%%%%%%%%%%
\begin{exercise}[]{Prove that for any integer $n^2 -1\;>\;3$, there is a prime $p$ satisfying $n!\;>\;p\;> n$}
  \begin{proof}
    Consider $n!-1$.
    
    Since $n!$ is the product of $1,2,\ldots,n$, $n!-1$ can't be divided by $1,2,\ldots,n$ with a remainder of 0.

    If $n!-1$ is a prime, then the result follows.
    
    If $n!-1$ is not a prime, then we can always find a prime divider greater than $n$,
    
    i.e. there is a prime $p$ satisfying $n! > p > n$
  \end{proof}
  \label{ex2}
\end{exercise}


%%%%%%%%%%%%%%%%%%%%%%%%%%%%%%%%%%%%%%%%%%
%%%%%%%%%%%%%                 %%%%%%%%%%%%
%%%%%%%%%%%%%    EXERCISE 3   %%%%%%%%%%%%
%%%%%%%%%%%%%                 %%%%%%%%%%%%
%%%%%%%%%%%%%%%%%%%%%%%%%%%%%%%%%%%%%%%%%%
\begin{exercise}[]{Assume that there is a recurrence formula as follows: 
  \begin{equation*}
    D(x) = \begin{cases}
    1, &if\;\lfloor x \rfloor \leq 1\\
    3D(x/4) + x - 2, &if\;\lfloor x \rfloor  > 1
    \end{cases}
  \end{equation*}
  Please deduce the non-recursive expression of $D(x)$ and point out its asymptotic complexity.    }
  \begin{solution}
  For $x\in \left(-\infty,2\right)$, $D(x)=1$

  For $x\in \left[2,8 \right)$, Let $D_{0}(x)=3D(x/4) + x - 2 = 3 + x -2 = x + 1$, $\ldots$

  For $x\in \left[2\cdot 4^{k}, 2\cdot 4^{k+1}\right)$, Let $D_{k}(x) = a_k x + b_k$, then
  \begin{equation}
    \begin{aligned}
      D_{k+1}(x) &= 3D_k (\frac{x}{4}) + x - 2 \\
      &= 3 \left(a_k \cdot \frac{x}{4} + b_k \right) + x - 2 \\
      &= \left(\frac{3}{4}a_k + 1 \right)x + (3b_k - 2)
    \end{aligned}
  \end{equation}
  It follows that
  \begin{equation}\left\{
    \begin{array}{l}
      a_{k+1} = \frac{3}{4}a_k + 1 \\
      b_{k+1} = 3b_k - 2 \\
    \end{array}\right. \implies \left\{
      \begin{array}{l}
        a_{k} - 4 = \frac{3}{4} (a_{k-1} -4) = \left(\frac{3}{4}\right)^{k-1} (-3)\\
        b_{k} - 1 = 3(b_{k-1} - 1) = 0 \\
      \end{array}\right. \implies \left\{
        \begin{array}{l}
          a_{k}  = 4 - \left(\frac{3}{4}\right)^{k-1} \cdot 3 \\
          b_{k} = 1 \\
        \end{array}\right.
  \end{equation}

  \begin{equation}
    \begin{aligned}
      D(x) &=  a_k x + b_k = \left(4 - \left(\frac{3}{4}\right)^{k-1} \cdot 3\right) x + 1 \text{ for } k \in \left( \log_4 2x -1, \log_4 2x \right] \\
      &= \left \{  
        \begin{array}{ll}
          \left(4 - 3 \left(\frac{3}{4}\right)^{ \lfloor \log_4 {2x}\rfloor-1}\right) x + 1 & x \ge 2 \\
          1 & x < 2
        \end{array}
      \right.
    \end{aligned}
  \end{equation}
  
  When $x$ is sufficiently large, $\left(4 - 3 \left(\frac{3}{4}\right)^{ \lfloor \log_4 {2x}\rfloor-1}\right) \rightarrow 4$, the asymptotic complexity is $O(x)$



  \end{solution}
  \label{ex3}
\end{exercise}


%%%%%%%%%%%%%%%%%%%%%%%%%%%%%%%%%%%%%%%%%%
%%%%%%%%%%%%%                 %%%%%%%%%%%%
%%%%%%%%%%%%%    EXERCISE 4   %%%%%%%%%%%%
%%%%%%%%%%%%%                 %%%%%%%%%%%%
%%%%%%%%%%%%%%%%%%%%%%%%%%%%%%%%%%%%%%%%%%
\begin{exercise}[]{Use the minimal counterexample principle to prove that for any integer $n > 10$, there exist integers $i_n\geq0$ and $j_n\geq 0$, such that $n = i_n \times 3 + j_n \times 4$.}
  \begin{proof}
     First, it is easy to check 6, 7, 8, 9, 10 can be written as some combination of 3 and 4.

     Suppose otherwise, then there exists a minimal counterexample $n'>10$ and $n'$ can't be written as the combination of 3 and 4. Since $\text{gcd}(3,4)=1$,  $n' - 3$ can't neither be written as the combination of 3 and 4.
     
     If $n' - 3 \le 10$, it contradicts to the previous checked fact. If $n' -3 > 10$, it contradicts to the assumption that $n'$ is the minimal counterexample.
  \end{proof}
  \label{ex4}
\end{exercise}


%%%%%%%%%%%%%%%%%%%%%%%%%%%%%%%%%%%%%%%%%%
%%%%%%%%%%%%%                 %%%%%%%%%%%%
%%%%%%%%%%%%%    EXERCISE 5   %%%%%%%%%%%%
%%%%%%%%%%%%%                 %%%%%%%%%%%%
%%%%%%%%%%%%%%%%%%%%%%%%%%%%%%%%%%%%%%%%%%
\begin{exercise}[]{Analyze the \textbf{average} time complexity of QuickSort in Alg.~\ref{Alg_Quick}.

  \begin{minipage}[t]{0.8\textwidth}
  \begin{algorithm}[H]
    \KwIn{An array $A[1,\cdots,n]$}
    \KwOut{$A[1,\cdots,n]$ sorted nondecreasingly}

    \BlankLine
    \caption{QuickSort}\label{Alg_Quick}

    \If{$n \le 1$}{
     \Return\;
    }

    $pivot \leftarrow A[n]$; $i \leftarrow 1$\;
    \For{$j \leftarrow 1$ \KwTo $n-1$}{
      \If{$A[j] < pivot$}{
        swap $A[i]$ and $A[j]$\;
        $i \leftarrow i+1$\;
      }
    }

    swap $A[i]$ and $A[n]$\;
    \lIf{$i>1$}{$\operatorname{QuickSort}(A[1,\cdots,i-1])$}
    \lIf{$i<n$}{$\operatorname{QuickSort}(A[i+1,\cdots,n])$}
  \end{algorithm}
  \end{minipage}
  }
  \begin{solution}
  For input of the size $N$, the loops from line 5 to 10 takes $O(N)$ time, and the recursion will take $T(i-1)$ and $T(N-i-1)$ time for elements smaller than the pivot and greater than the pivot, where $T(n)$ is the time taken for $n$ inputs.

  For average case, we can take the average of every possible dividing case. The partition of $N$ elements can be $\left\{0,N-1\right\}$, $\left\{1,N-2\right\}$, $\ldots$ , $\left\{N-1,0\right\}$. Hence we have
  \begin{equation}
      T(N) = 2\left(\frac{T(0)+T(1)+\ldots+T(N-1)}{N}\right) + cN
  \end{equation}
  \begin{equation}
    NT(N) = 2\left({T(0)+T(1)+\ldots+T(N-1)}\right) + cN^2
    \label{eqn1}
  \end{equation}
  \begin{equation}
    (N-1)T(N-1) = 2\left({T(0)+T(1)+\ldots+T(N-2)}\right) + c(N-1)^2
    \label{eqn2}
  \end{equation}
  By subtracting Equation \ref{eqn2} from Equation \ref{eqn1} we have
  \begin{equation}
    NT(N) - (N-1)T(N-1) = 2T(N-1) + 2cN - c^2
  \end{equation}
  Since constant $c^2$ can be ignored
  \begin{equation}
    \begin{aligned}
      NT(N) &= (N+1)T(N-1) +2cN \\
      \frac{T(N)}{N+1} &= \frac{T(N-1)}{N} + \frac{2c}{N+1} \\
      \frac{T(N-1)}{N} &= \frac{T(N-2)}{N-1} + \frac{2c}{N} \\
      \ldots & \ldots \\
      \frac{T(2)}{3} &= \frac{T(1)}{2} + \frac{2c}{3}
    \end{aligned}
    \label{eqn3}
  \end{equation}
  By taking the sum of Equation \ref{eqn3} together, we have
  \begin{equation}
    \begin{aligned}
      T(N) &= (N+1)\left(\frac{T(1)}{2} + 2c \sum_{i=3}^{N+1}\frac{1}{i}\right) \\
      &\le (N+1)2c \ln N \in O(N\log N)
    \end{aligned}
  \end{equation}
  \end{solution}
  \label{ex5}
\end{exercise}



%%%%%%%%%%%%%%%%%%%%%%%%%%%%%%%%%%%%%%%%%%
%%%%%%%%%%%%%                 %%%%%%%%%%%%
%%%%%%%%%%%%%    EXERCISE 6   %%%%%%%%%%%%
%%%%%%%%%%%%%                 %%%%%%%%%%%%
%%%%%%%%%%%%%%%%%%%%%%%%%%%%%%%%%%%%%%%%%%
\begin{exercise}[]{Rank the following functions by order of growth with explanations: that is, find an arrangement $g_1, g_2, \ldots , g_{k}$ of the functions $g_1 = \Omega(g_2), g_2 = \Omega(g_3), \ldots, g_{k-1} = \Omega(g_{k})$.  Partition your list into equivalence classes such that functions $f(n)$ and $g(n)$ are in the same class if and only if $f(n) = \Theta(g(n))$. Use symbols ``$=$'' and ``$\prec$'' to order these functions appropriately. (ps: $\log n$ refers to $\log_2 n$.)$$
  \begin{array}{ccccc}
      2^{\log n} \quad & \quad (\log n)^{\ln n} \quad & \quad n^2 \quad & \quad n! \quad & \quad (n - 1)! \\
      2^n \quad & \quad n^3 \quad & \quad \log^2 n \quad & \quad e^n \quad & \quad 2^{2^n} \\
      \log\log n \quad & \quad (n+1)\cdot 2^n \quad & \quad n \quad & \quad \log {(n^2 - n)} \quad & \quad 2^{\ln n} \\
  \end{array}
  $$}
  \begin{solution}
  \begin{equation}
    \begin{aligned}
      \log\log n  \prec \log {(n^2 - n)} \prec \log^2 n \prec 2^{\log n} = n = 2^{\ln n}  \prec n^2 \prec n^3 \\ 
      \prec (\log n)^{\ln n} \prec 2^n = e^n \prec (n+1)\cdot 2^n \prec  (n-1)! \prec n! \prec  2^{2^n} 
    \end{aligned}
  \end{equation}
  \end{solution}
  \label{ex6}
\end{exercise}

