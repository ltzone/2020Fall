\section{Problem Formulation}

Based on the previous metrics, we can get the formal definition of fairness assured online recommendation problem for services with capacity constraints.

\begin{definition}[Fairness Assured Recommendation Online Problem for Services with Capacity Constraints.]

\textbf{Given}, Relevance rating matrix $R$ and original recommendation lists $\{L\}$ generated by a conventional recommendation algorithm, the set of services' capacity constraint $C$, a positive integer $N$, an set of ordered lists of recommendation requests $Q$ where $q_{ij}$ is from a set of users $U$, indicating the $j^{\text{th}}$ request in $i^{\text{th}}$ round. 


\textbf{Objective}, Finding a top-$N$ recommendation list for each request to minimize $D(F_i^T)$, where $D\left(F_{i}^{T}\right)$ represents the variance among Top-N Fairness of all users.

\textbf{Constraint},
\begin{enumerate}\setlength{\itemsep}{-0.1cm}
    \item Capacity: keeping $\forall c_{j} \in C, c_{j} \geqslant \sum_{u_{i} \in U} \delta_{i}^{T} \cdot  \operatorname{Is\_In}\left(s_{j}, l_{i}^{T}, N\right)$
    \item Recommendation Quality: preserving $\sum_{u_{i} \in U} q_{i}^{T}$ at a high level
    \item Online Property: The recommendation for $q_{ij}$ should not be affected by $q_{i'j'}$ where $ i' > i$ or  $j'> j \And i' = i$ .
\end{enumerate}

\end{definition}

The improvement in our formulation compared with the original FAST system is that our problem regulates that the requests should be taken sequentially. Our problem also takes a user coming probability $A$ instead of the fixed user set as input. These two differences are what we believe to be of the greatest help for online implementation.
